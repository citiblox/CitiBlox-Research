\documentclass[conference]{IEEEtran}
\usepackage{authblk}
\usepackage{graphicx}
\graphicspath{ {./images/} }
%\usepackage{babel}
\usepackage{amsmath, amsfonts}

\usepackage[nottoc]{tocbibind}
\usepackage[european resistors, straightvoltages]{circuitikz}

\author[1]{Jay Piamjariyakul}
%\author[2]{Felix Kong}
\affil[1]{Undergraduate, Department of Electrical \& Electronic Engineering, University of Bristol}
%\affil[2]{Department of Mechanical Engineering, University of Bristol}
\title{Hot \& Motored: Generating Electricity with Footsteps}

\renewcommand\IEEEkeywordsname{Keywords}
\newcommand{\figref}[1]{\figurename~\ref{#1}}
\newcommand{\compref}[1]{Component~\ref{#1}}
\newcommand{\stepref}[1]{Step~\ref{#1}}

\begin{document}

\maketitle
\begin{abstract}
This document examines the plausibility \& efficiency of generating power via footsteps, which could be used on pavements to generate electricity used for various purposes, depending on the user's preferences.
\\
Excuse the pun.
\end{abstract}

% \begin{IEEEkeywords}
%     Component (comp.)
% \end{IEEEkeywords}
 
% \section{Preamble}
% The application of utilising footsteps to generate electricity adapts the mechanisms of a dynamo (which generates electricity from mechanical torque input) and that of a pressure plate (akin to those used in vintage spinning-top toys). Its applications, theoretically, could be used to alleviate lack of electricity in remote regions (given a large amount of foot-based commuting) or help local businesses collect data in regards to visitors/browsers \& determine their best course of business action.

\section{Mechanism Proposal}
The proposed mechanism to generate electricity is comprised of multiple stages, and such operate in unison to provide the output voltage at the output stage.

\subsection{Components Analysis}
The mechanism itself involves a button/plate that, when pressed, would push a spiral down an entry point, compressing a spring. When such button/plate (thus, the spring) is released, the spiral is released to its initial position, in turn spinning the flywheel disc/gear that such spiral is fixed to, causing such to move in a rotational manner.

\begin{figure}[ht]
    \centering
    \caption{Diagram of proposed mechanism to generate electricity from footsteps}
    \label{fig:mech_wlabel}
    \includegraphics[width=\linewidth]{mech_wlabel.png}
\end{figure}

Such mechanism is shown in \figref{fig:mech_wlabel} and is comprised of the following components (refer to labels on \figref{fig:mech_wlabel}):
\begin{enumerate}
    \item\label{item:comp:plate} Plate which individual steps onto - such is supported by a compression spring
    \item\label{item:comp:spiral} Spiral rod which inserts into \compref{item:comp:flywheel}'s axle - such is attached to \compref{item:comp:plate}
    \item\label{item:comp:flywheel} Flywheel which \compref{item:comp:spiral} inserts into via an opening fitting the size of the spiral's rectangular body
    \item\label{item:comp:motor} Motor with additional gears connected to \compref{item:comp:flywheel}, where such is repurposed as a generator; such gears may include transmission stages
\end{enumerate}

\subsection{Operations of Mechanism}
The mechanism is intended to operate per the following procedures (refer to labels on \figref{fig:mech_wlabel}):
\begin{enumerate}
    \item\label{item:op:stepdn} One steps onto \compref{item:comp:plate}, pushing \compref{item:comp:spiral} downwards \& compressing the support spring
    \item\label{item:op:spiraldn} \compref{item:comp:spiral} moving downwards passes through slot in \compref{item:comp:flywheel}
    \item\label{item:op:stepup} One steps away from \compref{item:comp:plate}, resulting in the spring being released; such results in \compref{item:comp:spiral} moving upwards to its original position
    \item\label{item:op:spin} \compref{item:comp:spiral} moving upwards now results in \compref{item:comp:flywheel} rotating, causing \compref{item:comp:motor}'s axle to also be rotated (due to the gear connection)
    \item\label{item:op:induce} The rotation of \compref{item:comp:motor}'s axle results in EMF being generated at its output
\end{enumerate}

One can surmise that power from \textbf{linear motion} (due to Steps \ref{item:op:stepdn} and \ref{item:op:spiraldn}) is converted to that of \textbf{rotary motion} (given by Step \ref{item:op:spin}), and later converted via \textbf{induction} (per Step \ref{item:op:induce}).

% ------------------------------
% ------------------------------

% \subsection{Electromagnetic Induction}
% One recalls Faraday's \textbf{Law of Electromagnetic Induction} (and Lenz' Law)
% \begin{equation}
    % % Use \varepsilon for curly epsilon
    % E = -\frac{d\psi}{dt} \equiv -N\frac{d\phi}{dt} = -N\frac{dBA}{dt}
    % \label{eq:faraday}
% \end{equation}
% where such terms are defined per following:
% \begin{itemize}
    % \item \(E\): \textbf{EMF} generated within coil (V)
    % \item \(\psi\): magnetic \textbf{flux linkage} (Wb Turns)
    % \item \(\phi\): magnetic \textbf{flux} (Wb)
    % \item \(N\): amount of wire turns within a coil (Turns)
    % \item \(B\): magnetic flux density (T)
    % \item \(A\): area of coil (m\(^2\))
% \end{itemize}

% Per \eqref{eq:faraday}, one can observe (with reference to \figref{fig:mechanism_side_wlabel} \& \figref{fig:mechanism_winding_top}) that, as the flywheel (Comp. 5) rotates, the magnetic field generated by the magnets (Comp. 6) only crosses through the wire coil windings, whereas the coil's area does not get altered, and thus is stationary. \eqref{eq:faraday} could thereby be simplified to the following (assuming coil is immute during operation):
% \begin{displaymath}
    % E = -NA\frac{dB}{dt}
% \end{displaymath}

% \subsection{Work Done by Footstep}
% One recalls the general formula of \textbf{power}:
% \begin{equation}
    % \label{eq:power:general}
    % P = \frac{\Delta W}{\Delta t}
% \end{equation}
% where \(P\) is power (measured in Watts), and \(W\) is \textbf{work done} (measured in Joules). Such is applicable in both \textbf{linear} \& \textbf{rotary} motions' stages of the operation.

% One can also recall the formula of \textbf{work done} via \textbf{linear motion}:
% \begin{equation}
    % W = F_L d = (F_f + F_k + F_c)(s-x)
    % \label{eq:work:linear}
% \end{equation}
% where:
% \begin{itemize}
    % \item \(F_L\): \textbf{aggregate force} exerted onto spiral, due to \textbf{linear motion} via person stepping on plate \& spring and damper (N)
    % \item \(F_f\): \textbf{force} exerted by \textbf{person} onto plate (N)
    % \item \(F_k\) \& \(F_c\): \textbf{forces} exerted onto plate by \textbf{spring} and \textbf{damper}, respectively; such forces oppose \(F_f\) \& are in opposite direction to \(F_f\), and thus are negative (N)
    % \item \(d\): maximum possible \textbf{displacement} which spiral/plate can travel downwards (m)
    % \item \(s\): \textbf{distance} between plate \& disc (m)
    % \item \(x\): minimum possible \textbf{displacement} of compressed spring \& damper, assuming both damper and spring compresses to the same displacement (m)
% \end{itemize}

% Similarly, one recalls the formula of \textbf{work done} via \textbf{rotational motion}:
% \begin{equation}
%     W = \tau\theta = F_R r\theta
%     \label{eq:work:rotational}
% \end{equation}
% where:
% \begin{itemize}
%     \item \(\tau\): \textbf{torque} of disc (Nm)
%     \item \(\theta\): \textbf{angle} of rotation of disc (radians)
%     \item \(F_R\): \textbf{force} of rotation tangential to disc, as exerted by spiral (N)
%     \item \(r\): \textbf{radius} of disc (m)
% \end{itemize}

% Per \textbf{conservation of energy}, such \textbf{cannot be created nor destroyed}. Given so, and assuming the following (amongst others):
% \begin{itemize}
%     \item System between linear \& rotational motion operates at 100\% efficiency (not possible in reality)
%     \item There is minimal friction between spiral \& the hole in the disc that such spiral enter into
%     \item Spring \& damper both compress to the same displacement \& returns to its default state
%     \item One stepping on \& off to/from the plate is reminiscent to a step input
% \end{itemize}
% one can equate \eqref{eq:work:linear} and \eqref{eq:work:rotational} to obtain the following:
% \begin{equation}
%     F_L d = F_R r\theta
% \end{equation}


\section{Theory}
% Insert brief intro of section
This section concerns the generator's output characteristics \& how such is linked to the mechanical stage prior. Such refers to \cite{industrial2} in regards to theory \& formula behind such operations.

%Generating electricity from footsteps adapts the fundamental theorems of mechatronics, prominent those involving electromagnetic induction. See \cite{industrial2} for reference.

%\subsection{Electromagnetic Induction}
% One recalls Faraday's \textbf{Law of Electromagnetic Induction} (and Lenz' Law)
% \begin{equation}
%     % Use \varepsilon for curly epsilon
%     \epsilon = -\frac{d\psi}{dt} \equiv -N\frac{d\phi}{dt} = -N\frac{dBA}{dt}
%     \label{eq:faraday}
% \end{equation}
% where such terms are defined per following:
% \begin{itemize}
%     \item \(\epsilon\): \textbf{EMF} generated within coil (V)
%     \item \(\psi\): magnetic \textbf{flux linkage} (Wb Turns)
%     \item \(\phi\): magnetic \textbf{flux} (Wb)
%     \item \(N\): amount of wire turns within a coil (Turns)
%     \item \(B\): magnetic flux density (T)
%     \item \(A\): area of coil (m\(^2\))
% \end{itemize}

% \begin{displaymath}
%     \epsilon = -NA\frac{dB}{dt}
% \end{displaymath}

\subsection{Assumptions of System}
Given reality, that the system will encounter losses, and thus will not be 100\% efficient. This therefore warrants for a number of assumptions to be made, to allow easier calculations \& analyses of the system.
\begin{itemize}
    \item An individual steps onto the slab \& pushes the spiral to its maximum length
    \item There is no friction within the spiral - this results in the flywheel rotation a full revolution per one successful spiral turn
    \item Thermal losses (due to friction or thermal dissipation by the components) are neglegible
    \item The material themselves do not wear down or permanenently change form, thus allowing for sustained usage
    \item The spiral's upward motion (and thus the flywheel's rotation speed) is constant
    \item The torque \& angular speed due to the flywheel is fully transferred to the motor/generator
\end{itemize}

\subsection{Known Quantities}
The following electrical values are either known or determined by the chosen equipment.
\begin{itemize}
    \item The \textbf{electromagnetic constant}, \textbf{reluctance} and \textbf{voltage rating} of the motor/generator
    \item Desired voltage at load (i.e. 5V for a rechargeable battery)
\end{itemize}
Similarly, on the mechanical domain one understands that the number of \textbf{spiral turns} is controlled by the spiral rod, and thus is a known value.

\subsection{Generator Output Characteristics}
The generator's output mechanism can be modelled as an equivalent circuit per \figref{circuit:generator}, where the following terms are defined:
\begin{itemize}
    \item $\omega$/$T$: Angular speed \& torque of generator's input axle, dictated by its transmission gear \& stages
    \item $E$: Output EMF of generator
    % \item $\mathcal{R}_a$: Generator reluctance
    \item $R_a$: Generator reluctance
    \item $I_a$/$V_a$: Current \& voltage at load output of generator
\end{itemize}

\begin{figure}[ht]
    \centering
    \label{circuit:generator}
    \caption{Equivalent circuit of generator and its output}
    \begin{circuitikz}
        \draw (0, 0) to[Telmech=G, n=generator] (0, 2) to[short, -*] (1, 2);
        \draw (1, 2) to[R=\text{\(R_a\)}, i=\(I_a\), -o] (4,2);
        \draw (1, 0) to [open, v>=$E$] (1, 2);
        \draw (0, 0) to[short, -*] (1, 0) to[short, -o] (4, 0);
        \draw (4, 0) to [open, v>=$V_{a}$] (4, 2);
        \draw [thick, ->>] (-1.5, 1) -- (generator.left) node[
            midway,above]{$\omega$} node[
                midway,below]{$T$};
    \end{circuitikz}
\end{figure}

Such is governed by the equations given by \eqref{eq:dcmotor}, where $k_e$ is the \textbf{electromagnetic constant} of the motor/generator.
\begin{subequations}
    \label{eq:dcmotor}
    \begin{align}
        T &= k_e I_a
        \label{eq:dcmotor:torque}\\
        E &= k_e \omega
        \label{eq:dcmotor:emf}
    \end{align}
\end{subequations}

Given \figref{circuit:generator} one can surmise the following information given by \eqref{eq:dcmotor}.
\begin{equation}
    \label{eq:circuit}
    \begin{aligned}
        V_a &= E - I_aR_a\\
        &= k_e\omega - \left(\frac{T}{k_e}\right)R_a
    \end{aligned}
\end{equation}
where \(\omega\) and \(T\) are determined by the prior stage, specifically the mechanical input stage.

\subsection{Mechanical Input Characteristics}
% Calculate general omega/tau from spiral rod!!
\begin{subequations}
    For one spiral turn
    \begin{align}
        \omega &= \frac{n 2\pi r}{\Delta t}
    \end{align}
\end{subequations}


% \section{Simulation}
% One utilises MATLAB's SimScape to simulate such mechanisms, and is comprised of three stages.
% % Check if intro line is necessary and/or needs amendment!

% \subsection{Schematic Diagram}

% ------------------------------
% ------------------------------

\medskip
\bibliographystyle{IEEEtran}
\bibliography{references}
\end{document}